\documentclass[letterpaper,11pt]{article}

\usepackage[shortlabels]{enumitem}
\usepackage[margin=1in]{geometry}
\usepackage[most]{tcolorbox}

\begin{document}
\title{{\bf Unit 3: Basic Neuroanatomy Project} }
\author{Name: Lucy}

\date{}
\maketitle

\section*{Short Answer}
\begin{enumerate}[a)]
\item Describe several advantages and disadvantages of biological computation with the brain compared to machine learning

\begin{tcolorbox}
Biological computation in the brain is highly energy-efficient, adaptive, and capable of learning from limited examples, but it’s also slow, noisy, and difficult to control or interpret. In contrast, machine learning systems seem to be able to process vast amounts of data quickly and precisely, with consistent and measurable results. However, they require large datasets, consume far more energy, and lack the flexibility, intuition, and contextual understanding of the human brain. In short, the brain learns efficiently but imprecisely, while machine learning learns precisely but inflexibly.
\end{tcolorbox}

\item Speculate what aspects of the architecture of the brain may cause these advantages or disadvantages, and similarly comment on aspects of machine learning’s architecture

\begin{tcolorbox}
The brain’s advantages mainly come from its plasticity and adaptibility. The dynamic network system connected by the synapses are strengthened or weakened by experience, allowing us to learn with flexibility and adapt to new environments really easily. This is what the current machine learning models are lacking; although they are extremely fast and clear unlike the brain based on the perfect mathematical structure, it has fixed rules and design that cannot be easily altered (it's definitely getting better though!!). In summary, the brain’s architecture prioritizes adaptation and survival, while machine learning’s prioritizes optimization and control.
\end{tcolorbox}

\item Brainstorm some marvelous schemes for integrating advantages from both ways of computing. Draw, write, scribble etc… When you are done, do a quick google for your best ideas to see if anyone has researched or tried them already!

\begin{tcolorbox}
Well, not my original idea. But I want to introduce you to the idea of "Multi-Plasticity Network (MPN)" by Kyle Aitken from the Allen Institute. I read this paper, which was published in February 2023, in my current seminar on computational neuroscience. This new model suggests a way of performing computation through dynamically modulating synaptic strengths, without needing recurrent loops. It's really interesting. You should search it up!
\end{tcolorbox}
    
\end{enumerate}
\end{document}